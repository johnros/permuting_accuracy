\documentclass[12pt,a4paper]{article}
\usepackage[utf8]{inputenc}
\usepackage{amsmath}
\usepackage{amsfonts}
\usepackage{amssymb}
\usepackage{graphicx}
\usepackage{todonotes}
\usepackage{natbib}
\usepackage{url}
\usepackage[boxruled,vlined,linesnumbered]{algorithm2e}
\usepackage{caption}
\usepackage{subcaption}
\usepackage{lineno}


\author{Jonathan Rosenblatt \and Roee Gilron \and Roy Mukamel}


%% OPTIONAL MACRO DEFINITIONS
%\def\s{\sigma}
\newcommand{\set}[1]{\{ #1 \}} % A set
\newcommand{\indicator}[1]{I_{\set{#1}}} % The indicator function.
\newcommand{\reals}{\mathbb{R}} % the set of real numbers
\newcommand{\features}{x} % The feature space
\newcommand{\outcomes}{y} % The feature space
\newcommand{\featureS}{\mathcal{X}} % The feature space
\newcommand{\outcomeS}{\mathcal{Y}} % The feature space
\newcommand{\hyp}{f} % A hypothesis
\newcommand{\hypEstim}{\hat{\hyp}} % A hypothesis
\newcommand{\hypclass}{\mathcal{F}}
\newcommand{\expect}[1]{\mathbf{E}\left[ #1 \right]} % The expectation operator
\newcommand{\acc}{T^{acc}} 
\newcommand{\dominant}{\hat{p}_{max}}
\newcommand{\prob}[1]{Prob( #1 )} % the probability of an event






\title{Better-Than-Chance Classification for Signal Detection}


\begin{document}
%Sketch: 
%- The conservativeness of the test. 
%- What does it detect?
%- Is it remedied in large samples?
%- Is conservativeness always there?


\maketitle
\linenumbers

\begin{abstract}
[TODO]
\end{abstract}


%%%% Introduction %%%
\section{Introduction}
\label{sec:introduction}

A common workflow in neuroimaging consists of fitting a classifier, and estimating its predictive accuracy using cross validation. 
Given that the cross validated accuracy is a random quantity, it is then common to test if the cross validated accuracy is significantly better than chance using a permutation test.  
Examples in the neuroscientific literature include \citet{golland_permutation_2003,pereira_machine_2009,varoquaux_assessing_2016}, and especially the recently populirized \emph{multivariate pattern analysis} (MVPA) framework of \citet{kriegeskorte_information-based_2006}.
This practice is also observed in the genetics literature, but to a lesser extent \citep{radmacher_paradigm_2002,jiang_calculating_2008}.

To fix ideas, we will adhere to a concrete example.
In \cite{gilron_quantifying_2016}, the authors seek to detect brain regions which encode differences between vocal and non-vocal stimuli. 
Following the MVPA workflow, the localization problem is cast as a supervised learning problem: if the type of the stimulus can be predicted from the spatial activation pattern significantly better than chance, then a region is declared to encode vocal/non-vocal information. 
We call this an \emph{accuracy test}, a.k.a.\ \emph{class prediction} in \cite{simon_pitfalls_2003}, or \emph{pattern discrimination} in \cite{pereira_machine_2009}.

This same signal detection task can be also approached as a two-group multivariate test.
Inferring that a region encodes vocal/non-vocal information, is essentially inferring that the spatial distribution of brain activations is different given a vocal/non-vocal stimulus. 
As put in \cite{pereira_machine_2009}: 
\begin{quote}
... the problem of deciding whether the classifier learned to discriminate the classes can be subsumed into the more general question as to whether there is evidence that the underlying distributions of each class are equal or not.
\end{quote}
A practitioner may then call upon a two-group location test such as Hotelling's $T^2$ \citep{anderson_introduction_2003}.
Alternatively, if the size of a brain region is too large compared to the number of observations, so that the spatial covariance cannot be fully estimated, then a high dimensional version of Hotelling's test can be called upon, such as in \cite{schafer_shrinkage_2005} or \cite{srivastava_testing_2013}.
For breivity, and in contrast to \emph{accuracy tests}, we will call these two-sample multivaraite tests simply \emph{location tests}, also termed \emph{class comparisons} in \cite{simon_pitfalls_2003}.

At this point, it becomes unclear which is preferable: a location test or an accuracy test?
The former with a heritage dating back to \cite{hotelling_generalization_1931}, and the latter being extremely popular, as the $959$ citations\footnote{GoogleScholar. Accesed on Aug 4, 2016.} of \cite{kriegeskorte_information-based_2006} suggest. 

The comparison between location and accuracy tests was precisely the goal of \cite{ramdas_classification_2016}, who compared the $T^2$ location test to the accuracy of \emph{Fisher's linear discriminant analysis} classifier (LDA). 
By comparing the rates of convergence of the powers to $1$, \cite{ramdas_classification_2016} concluded that accuracy and location tests are rate equivalent. 
Judging by convergence rates alone, not much is (asymptoticaly) lost by using an accuracy test. 
Asymptotic relative efficency measures (ARE) are typically used by statisticians to compare between test statistics with similar rates \citep{vaart_asymptotic_1998}.

The ARE between Hotelling's $T^2$ (location) test and Fisher's LDA (accuracy) test in \cite{ramdas_classification_2016} is lower bounded by $\sqrt{2 \pi} \approx 2.5$. 
This means that Fisher's LDA requires at least $2.5$ more samples to achieve the same (asymptotic) power than the $T^2$ test. 
In this light, the accuracy test is remarakbly inneficient compared to the location test.  
For comparison, the t-test is only $1.04$ more (asymptotically) efficienct than Wilcoxon's rank-sum test \citep{lehmann_parametric_2009}, so that an ARE of $2.5$ is strong evidence in favour of the location test. 

Before discarding accuracy tests, we recall that \cite{ramdas_classification_2016} analyzed a half-sample holdout. 
The authors thus conjecture that a leave-one-out approach, which makes more efficient use of the data, may have better performance. 
On the other hand, the analysis in \cite{ramdas_classification_2016} is asymptotic. This eschews the discrete nature of the accuracy statistic, which will shown to have  acrucial impact. 
Since typical sample sizes in neuroscience are not large, we seek to study which test is to be preferred in finite samples? 
Our conclusion will be quite simple: {\em location tests almost always have more power than accuracy tests}.

The main argument for our statement rests upon the observation that with typical sample sizes, the accuracy test statistic is highly discrete. 
Discrete test statistics are known to be conservative \citep{hemerik_exact_2014}, since they are insensitive to mild perturbations of the data, and they cannot exhaust the permissible false positive rate. 
The degree of discretization is governed by the number of samples. 
In our neuroscience example from \citep{gilron_quantifying_2016}, the classification is performed based on $40$ trials, so that the test statistic may assume only $40$ possible values. 
This number of examples is not unusual if considering this is the number of subjects, or the number of trial-repeats in an neuroimaging study. 

The discretization effect is aggravated if the test statistic is highly concentrated. 
For an intuition consider the usage of a the \emph{training} accuracy as a test statistic.
This is the \emph{resubstition classification} in \cite{ramdas_classification_2016}, and simply means that the accuracy is not cross validated. 
If the data is high dimensional, the train accuracy will be very high due to over fitting. 
In an extreme case, the train accuracy will be $1$ for the observed data, but also for any permutaiton. 
The concentration of the train accuracy near $1$, and its discreteness, render this test completely useless, with a power of $0$. 


To compare the power of accuracy tests and location tests in finite samples, we perform a simulation study of a battery of test statistics. 
The main findings are reported in Section~\ref{sec:power}, and the intution for our findings is provided in Section~\ref{sec:discussion}, but first, the problem's setup. 





%%%% Section %%%%
\section{Problem setup}
\label{sec:problem_setup}

Let $\outcomes \in \outcomeS$ be a class encoding. 
Let $\features \in \featureS$ be a $p$ dimensional feature vector. 
In our vocal/non-vocal example we have $\outcomeS=\set{-1,1}$ and $p$, the number of voxels in a brain region so that $\featureS=\reals^{27}$. 

Given $n$ pairs of $(\features_i,\outcomes_i)$, typically assumed i.i.d., a location test amounts to testing whether $\features|\outcomes=1$ has the the same distribution as $\features|\outcomes=-1$. 
I.e., we test if the multivariate voxel activation pattern has the same distribution when given a vocal stimulus, as when given a non-vocal stimulus. 
An accuracy test amounts to learning a predictive model $\hypEstim(\features)$ from some assumed model class $\hypEstim \in \hypclass$. 
The prediction accuracy, denoted $\acc_{\hypEstim}$, is defined as the probability of a given classifier $\hypEstim$ of making a correct prediction $\acc_{\hypEstim}:=\prob{\hypEstim(x)=y}$ when given a randomly drawn data point, ($\features,\outcomes$).
A statistically significant ``better than chance'' estimate of $\acc_{\hypEstim}$ is evidence that the classes are distinct. 


\subsection{Candidate Tests}
\label{sec:considerations}

The design of a permutation test using the prediction accuracy, requires the following design choices: 
\begin{enumerate}
\item How to estimate accuracy?
\item Is the statistic cross validated or not?
\item For a K-fold cross validated test statistic: should the data be refolded in each permutation? 
\item Permute labels of features?
\item For a K-fold cross validated test statistic: should the data folding balanced (a.k.a.\ stratified)?
\item How many folds? 
\end{enumerate}

We will now address these questions while bearing in mind that unlike the typical supervised learning setup, we are not interested in an unbiased estimate of the prediction error, but rather in the mere detection of a difference between two groups. 

\paragraph{How to estimate accuracy?}
\label{sec:estimate_accuracy}
Given a predictor $\hypEstim$, a natural test statistic is some estimate of its accuracy $\acc_{\hypEstim}$.
Complicating matters: very low accuracies, even $0$, is evidence that the classes are separated, and we only need to invert the predictions. 
We can thus consider $|\acc_{\hypEstim}-0.5|$ as the test statistic.
This, however, implies that if the classes are identical, random guessing has $0.5$ accuracy. This is not true if the classes are not balanced. 
The chance level in which case is the prevalence of the dominant class, we denote by $\dominant$.
This suggests the following test statistic $|\acc_{\hypEstim}-\dominant|$.
Since we will be aggregating these statistics over random data sets where the dominant class may have varying frequencies, it seems appropriate to standardize the scale of this statistic. 
We thus also consider the z-scored accuracy: $|\acc_{\hypEstim}-\dominant|/\sqrt{\dominant(1-\dominant)}$.


\paragraph{Cross validate or not?}
Were we interested in an unbiased estimator of the prediction error, there is no question that some independent validation is in order. 
Since we are merely interested in detecting a difference between classes, a biased error estimate is not an issue provided that bias is consistent over all permutations. 
The underlying intuition is that if the exact same computation is performed over all permutations, then a permutation test will be ``fair'', i.e., will not inflate the false positive rate. 
We will thus be considering both cross validated accuracies, and \emph{train} accuracies as our test statistics, a.k.a.\ \emph{resubstitution classification}. 


\paragraph{Refolding?}
The standard practice in neuroimaging is to refold the data after each permutation \citep{pereira_machine_2009}.
This is imperative if permuting labels while aiming at balanced data folds. 
This is not, however, imperative in general. 
For simplicity, we will adhere to the standard practice of refolding the data within each permutation.


\paragraph{Permute labels of features?}
While seemingly identical, the compounding of permutations with data foldings renders these two approaches distinct. 
As an example, consider balanced (stratified) K-fold cross validation where the initial data folding is balanced. 
After a label permutation, the original folds will probably not be balanced. 
If the \emph{features} are permuted, then the labels conserve their original fold assignments, and the original folds are balanced after each permutation. 
Since we only report results while refolding the data in each permutation, then the only difference between permuting labels and permuting features seems to be a computational one. 
We thus adhere to the more common, albeit computationally less efficient practice of permuting labels. 


\paragraph{Balanced folding?}
As already implied, a standard practice when cross validating is to constrain the data folds to be balanced (i.e. stratified).
This is well justified when aiming at unbiased accuracy estimation. 
This also simplifies matter when aiming at signal detection, as can be seen from the above discussion of the appropriate test statistic. 
On the other hand, it may complicate matters, as can be seen from the above discussion on label versus feature permutation. 
We will report results with both balanced and unbalanced data foldings, only to discover, it does not really matter. 


\paragraph{How many folds?}
Different authors suggest different rules for the number of folds. 
We will be varying the number of folds.
This will affect the concentration of permutation distribution of the estimated accuracy, which will have a crucial effect on the conservativeness of the accuracy test. 
Our intuition suggests that since more folds imply a less concentrated estimate, then leave-one-out should be the less conservative, and 2-fold should be the most conservative. 


The of tests we will be comparing is collected for convenience in Table~\ref{tab:collected}.


\begin{table}[th]
\centering
\begin{tabular}{l|c|c|c|c}
Name & Basis & CV & Accuracy & Parameters\\ 
\hline
\hline
Hotelling & Hotelling & -- & -- & shrink=FALSE\\ 
Hotelling.shrink & Hotelling & -- & -- & shrink=TRUE \\ 
lda.CV.1 & LDA & TRUE & accuracy &  -- \\ 
lda.CV.2 & LDA & TRUE & z-accuracy & -- \\ 
lda.noCV.1 & LDA & FALSE & accuracy &  --\\ 
lda.noCV.2 & LDA & FALSE & z-accuracy &  --\\ 
sd & SD & -- & -- & -- \\ 
svm.CV.1 & SVM & TRUE & accuracy & cost=1e1 \\ 
svm.CV.2 & SVM & TRUE & accuracy & cost=1e-1 \\ 
svm.CV.3 & SVM & TRUE & z-accuracy & cost=1e1 \\ 
svm.CV.4 & SVM & TRUE & z-accuracy & cost=1e-1 \\ 
svm.noCV.1 & SVM & FALSE & accuracy & cost=1e1 \\ 
svm.noCV.2 & SVM & FALSE & accuracy & cost=1e-1 \\ 
svm.noCV.3 & SVM & FALSE & z-accuracy & cost=1e1 \\ 
svm.noCV.4 & SVM & FALSE & z-accuracy & cost=1e-1 \\
\end{tabular} 
\caption{\footnotesize
This table enumerates the various test statistics we will be studying. 
Three are location tests: Hotelling, Hotelling.shrink, and sd.
\textit{Hotelling} is the classical two-group $T^2$ statistic. 
\textit{Hotelling.shrink} is a high dimensional version with the regularized covariance in \cite{schafer_shrinkage_2005}. 
\textit{sd} is another high dimensional version of the $T^2$, from \cite{srivastava_two_2013}. 
The rest of the tests are variations of the linear SVM, and Fisher's LDA, with varying accuracy measures, cross validated or not, and varying tuning parameters. 
For example, \textit{svm.CV.4} is a linear SVM, with \textit{libsvm}'s cost parameter set at $0.1$, using the cross validated z-scored accuracy ($|\acc_{\hypEstim}-\dominant|/\sqrt{\dominant(1-\dominant)}$, see Section~\ref{sec:considerations}).
Another example is \textit{lda.noCV.1}, which is Fisher's LDA, returning the train accuracy, without cross validation, and without z-scoring. 
}
\label{tab:collected}
\end{table}







%%%% Section %%%%
\section{Controlling the False Positive Rate}
\label{sec:type_i}

Figure~\ref{fig:simulation_1} demonstrates that all of the tests considered conserve the desired $0.05$ false positive rate, up to varying levels of conservativism.
This can be seen from the fact that the probability of rejection is no larger than $0.05$ in the abscense of any effect, encoded by a red circle. 
This is true, in particular if:
(a) the folds are balanced or not,
(b) the tuning parameters of some test statistic are varied,
(d) the number of folds is varied.
We also observe that the most conservative tests are the accuracy tests that are not cross validated. 
We return to this matter in the Discussion.




\begin{figure}[h]
\centering
\caption{\footnotesize
The power of a permutation test with various test statistics. 
The power on the $x$ axis. 
Effect are color and shape coded. They are assumed to be equal in all the $23$ dimensions, and vary over $0$ (red circle), $0.25$ (green triangle), and $0.5$ (blue square). 
The various statistics on the $y$ axis. Their details are given in Table~\ref{tab:collected}. 
Simulation code available at [TODO].}	
\label{fig:simulation_1}
\begin{subfigure}{.5\textwidth}
\centering
\includegraphics[width=1\linewidth]{"art/2016-07-26 20:55:48"}
\caption{Unbalanced.}  %TODO
\label{fig:simulation_11}
\end{subfigure}%
\begin{subfigure}{.5\textwidth}
\centering
\includegraphics[width=1\linewidth]{"art/2016-07-27 11:42:05"}
\caption{Balanced.} %TODO
\label{fig:simulation_12}
\end{subfigure}
\end{figure}




%%%% Section %%%%
\section{Power}
\label{sec:power}

Having established that all of the tests in our battery control the false positive rate, it remains to be seen if they have similar power-- 
especially when comparing the power of location tests to accuracy tests. 
From the simulation results reported in Appendix~\ref{apx:simulations} we collect the following insights:
\begin{enumerate}
\item Location tests have more power than accuracy tests in all our configurations.
\item The conservativness decays as the sample grows (Figure~\ref{fig:large_sample}), supporting the statement that discretization is responsible for power loss. 
\item The power is may increase or decrease with the number of folds (Figure~\ref{fig:n_folds}). [TODO:effect of n.folds.]
\item ... The z-scoring of the accuracies was introduced to deal with unbalanced foldings. 
If the z-scoring has any effect at all, it merely kills power.
There is really no reason to use it. 
\item ... [TODO: effect of balancing].
\item ... [TODO: heavy tails].
\item ... [TODO: signal in scale].
\item ... [TODO: correlation between voxels].
\end{enumerate}


The major insight from simulations is that the use of accuracy tests for signal detection is underpowered compared to location tests. 
We now verify this finding on a neuroimaging dataset.



\section{Neuroimaging Example}
\label{sec:example}
% Power of SVM versus SD.
% Difficulties of implementation.

Figure~\ref{fig:read_data} is an application of both a location and an accuracy test to the data of \cite{pernet_human_2015}. 
The authors of \cite{pernet_human_2015} collected fMRI data while subjects were exposed to the sounds of human speech (vocal), and other non-vocal sounds. 
Each subject was exposed to $20$ sounds of each type, totalling in $n=40$ trials in each scan.
The study was rather large and consisted of about $200$ subjects.
The data was kindly made available by the authors at the OpenfMRI website\footnote{\url{https://openfmri.org/}}.

We perform group inference using within-subjcet permutations using the pipeline of \cite{stelzer_statistical_2013}, which was also reported in \cite{gilron_quantifying_2016}. 
For completeness, the pipeline is described in Appendix~\ref{apx:analysis}. 
To demonstrate our point, we compare the \emph{sd} location test with the \emph{svm.cv.1} accuracy test (see Table~\ref{tab:collected} for the definition of these statistics). 

In agreement with our simulation results, the location test (\emph{sd}) discovers more brain regions when compared to an accuracy test (\emph{svm.cv.1}).
The former discovers $1,232$ regions, while the latter only $441$, as depicted in Figure~\ref{fig:read_data}.
We emphasize that both test statistics were compared with the same permutation scheme, and the same error controls, so that any difference in detections is due to their different power.

Having established that accuracy tests are underpowered both in simulation and in application, we wish to identify the conditions under which this will occur, and discuss implications on the practice of accuracy tests. 


\begin{figure}[th]
\centering
\includegraphics[width=0.7\linewidth]{"art/svm_vs_SD"}
\caption{\footnotesize
Brain regions encoding information discriminating between vocal and non-vocal stimuli.
Map reports the centres of $27$-voxel sized spherical regions, as discovered by an accuracy test (\emph{svm.cv.1}), and a location test (\emph{sd}). 
\emph{svm.cv.1} was computed using $5$-fold cross validation, and a cost parameter of $1$. 
Region-wise significance was determined using the permutation scheme of \cite{stelzer_statistical_2013}, followed by region-wise $FDR \leq 0.05$ control using the Benjamini-Hochberg procedure \citep{benjamini_controlling_1995}.
Number of permutations equals $400$.
The location test detect $1,232$ regions, and the accuracy test $441$, $399$ of which are common to both.
For the details of the analysis see Appendix~\ref{apx:analysis} and \cite{gilron_quantifying_2016}.  
  }
\label{fig:read_data}
\end{figure}








%%%% Section %%%%
\section{Discussion}
\label{sec:discussion}
% Not all problems are signal detection.
% Implementation difficulty with discrete test statistics.
% Signal not in location.
% Heavy tails.


We have set out to understand which of the tests is more powerful: the accuracy test or the location test. 
Using simulations, we have concluded that the location tests are preferable. 
We attribute this to several phenomena: 
(a) Discretization introduced in finite samples by the accuracy test statistic. 
(b) Inneficient use of the data for the validation set. 

The sensitivity of the power to the number of folds suggests that most of the power is lost due to the discretization and not to the holdout. 
The degree of discretization is govenred by the sample size. 
For this reason, an asymptotic analysis such as \cite{ramdas_classification_2016} may uncover the holdout inneficiency, but will not uncover the discretization effect. 
The practical advice for the practitioner, is that for the purpose of signal detection, there is typically a multivaraite test (be it a location test or other), that is more powerful. There is also a good chance that it would be easier to imeplement, since no validation will be involved. 

\paragraph{Neyman-Pearson Learning}
[TODO: optimizing type I or type II errors].
\cite{scott_neyman-pearson_2005}

\paragraph{A good accuracy test}
[TODO: discuss other findings in the power section]

\paragraph{Related Literature}
\cite{olivetti_induction_2012} and \cite{olivetti_statistical_2014} also looked into a similar problem as we do, namely, what is the preferred accuracy test?
They propose a new test they call an \emph{independence test}, and demonstrate by simulation that it has more power than other accuracy tests, and can deal with non-balanced data sets. 
We did not include this test in the battery we compared, but we note the following: 
(a) The independence test of \cite{olivetti_induction_2012} relies on a discrete test statistic. This means that in the cases that the accuracy test is called upon for discriminating populations, it will probably be underpowered compared to location tests. 
(b) The problem of the accuracy test with unbalanced data-sets, which motivates \cite{olivetti_induction_2012}'s independence test, can also be remedied by replacing the accuracy statistic with its z-score, as suggested in Section~\ref{sec:estimate_accuracy}.

\paragraph{Reservations}
At this point some reservations to the generality of our findings are in order. 
Firstly, not all accuracy tests are concerned with signal detection.
Indeed, it is possible that the purpose of the test is not to detect a difference between classes, but to actually test is a particular classifier is better than chance. 
This would be the case in decoding applications, like brain-machine interfaces, where the localization a signal is not enough. 
Clinical diagnosis is another application, where the presence of a medical condition is ``predicted'' from imaging data. \citep[e.g.][]{olivetti_induction_2012,wager_fmri-based_2013}

Secondly, not all signals are manifested in a shift of the null distrubiton. 
Put differently, the preferred alternative to an accuracy test is not always a location test. 
Indeed, one may consider signal, i.e. effects, as a change in scale, such as the \emph{spiked covariance} model. In this case, other-than-Hotelling type tests are appropriate~[TODO: cite change in covariance alternative].
Tests have been proposed even when the nature of the difference between populations is left unspecified \citep[e.g.][]{gretton_kernel_2012}.
The fact that in our neuroimaging example (Section~\ref{sec:example}) some brain regions were detected with the accuracy test, and not the location test, is consistent with this observation. 
On the other hand, the far greater power of the location test, certianly in our example, does serve as en empirical evidence that changes in location are a prevalent phenomenon. 
[TODO: signal in scale? heavy tails?]

\paragraph{Ease of implementation}
A very important point is the ease of implementation. 
The need for cross validation of the accuracy test greatly increases its computational complexity. 
Moreover, anyone who has actually implemented tests with discrete statistics, will attest they are considerably harder to implement. This is because their unforgiveness to the type of inequality. 
Indeed, mistakenly replacing a weak inequality with a strong inequality in one's program may considerably change the results. 
This is not the case for continuous test statistics. 

\paragraph{Epilogue}
Given all the above, we find the popularity of accuracy tests quite puzzling. 
We believe this is due to a reversal of the inference cascade. 
Researchers first fit a classifier, and then ask if the classes are any different.
Were they to start by asking if classes are any different, and only then try to classify, then location tests would naturally arise as the preferred method. 
As put by \cite{ramdas_classification_2016}:
\begin{quote}
The recent popularity of machine learning has resulted in the extensive teaching and use
of prediction in theoretical and applied communities and the relative lack of awareness or
popularity of the topic of Neyman-Pearson style hypothesis testing in the computer science
and related ``data science'' communities.
\end{quote}






%\begin{acknowledgments}
%TODO: 
% isf 900/60, Jelle, Jesse B.A. Hemerik, Yakir Brechenko, Omer Shamir
%\end{acknowledgments}












\bibliographystyle{abbrvnat}
\bibliography{Permuting_Accuracy.bib}

\appendix


\newpage

\section{Analysis pipeline}
\label{apx:analysis}

Here is the analysis pipeline of \cite{stelzer_statistical_2013} we for the auditory data in \cite{gilron_quantifying_2016}.
Denoting by 
$i=1,\dots,I$ the subject index, 
$v=1,\dots,V$ the voxel index, and 
$s = 1,\dots,S$ the permutation index. 
Since regions\footnote{\emph{searchlight} or \emph{sphere} in the MVPA parlance} are centred around a unique voxel, the voxel index $v$ also serves as a unique region index.
Algorithm~\ref{algo:statistic} computes a region-wise test statistic, which is compared to its permutation null distribution computed by Algorithm~\ref{algo:permutation}.


\begin{algorithm}[H]
\caption{Compute a group parametric map.}
\label{algo:statistic}

 \KwData{fMRI scans, and experimental design.}
 \KwResult{Brain map of group statistics: $\{\bar{T}_v\}_{v=1}^V$}
	 \For{$v \in 1,\dots,V$}{
		 \For{$i \in 1,\dots,I$}{
			 $T_{i,v} \leftarrow$ test statistic for subject $i$ in a region centered at $v$.
			 } 	  
	  	 $\bar{T}_{v} \leftarrow \frac{1}{I}\sum_{i=1}^I T_{i,v}$. 
 	 }
\end{algorithm}


\begin{algorithm}[H]
\caption{Compute a permutation p-value map.} 
\label{algo:permutation}

 \KwData{fMRI scans of $20$ subjects, experimental design.}
 \KwResult{Brain map of permutation p-values: $\{p_v\}_{v=1}^V$}
  \For{$s \in 1,\dots\,S$}{
    	    permute labels\;
    	    $\bar{T}_{v}^s \leftarrow$ parametric map 
  
  	}
\end{algorithm}

\newpage

\section{Simulations}
\label{apx:simulations}



\begin{figure}[h]
\centering
\caption{\footnotesize [TODO].}	
\label{fig:n_folds}
	\begin{subfigure}{.5\textwidth}
	  \centering
	  \includegraphics[width=1\linewidth]{"art/2016-07-27 21:21:12"}
	  \caption{2 Folds. Balanced}  %TODO
	\label{fig:n_folds_1}
	\end{subfigure}%
	\begin{subfigure}{.5\textwidth}
	  \centering
	  \includegraphics[width=1\linewidth]{"art/2016-07-29 07:18:24"}
	  \caption{20 Folds.Balanced} %TODO
	\label{fig:n_folds_2}
	\end{subfigure}
\end{figure}



\begin{figure}[h]
\centering
\caption{\footnotesize [TODO].}	
\label{fig:n_folds_unbalanced}
	\begin{subfigure}{.5\textwidth}
	  \centering
  	  \missingfigure[figwidth=6cm]{TODO}
%	  \includegraphics[width=1\linewidth]{"art/2016-07-27 21:21:12"}
	  \caption{2 Folds. Unalanced}  %TODO
	\label{fig:n_folds_unbalanced_1}
	\end{subfigure}%
	\begin{subfigure}{.5\textwidth}
	  \centering
	  \missingfigure[figwidth=6cm]{TODO}
%	  \includegraphics[width=1\linewidth]{"art/2016-07-29 07:18:24"}
	  \caption{20 Folds. Unbalanced} %TODO
	\label{fig:n_folds_unbalanced_2}
	\end{subfigure}
\end{figure}



\begin{figure}[h]
\centering
\caption{\footnotesize [TODO].}	
%\label{fig:simulation_1}
	\begin{subfigure}{.5\textwidth}
	  \centering
	  \includegraphics[width=1\linewidth]{"art/2016-07-30 10:33:05"}
	  \caption{Scale Change.}  %TODO
%	\label{fig:2016-07-2721:21:12}
	\end{subfigure}%
	\begin{subfigure}{.5\textwidth}
	  \centering
	  \missingfigure[figwidth=6cm]{TODO}
%	  \includegraphics[width=1\linewidth]{"art/2016-07-27 11:42:05"}
	  \caption{t Null} %TODO
%	\label{fig:2016-07-2721:21:12}
	\end{subfigure}
\end{figure}



\begin{figure}[h]
\centering
\caption{\footnotesize [TODO].}	
%\label{fig:simulation_1}
	\begin{subfigure}{.5\textwidth}
	  \centering
  	  \missingfigure[figwidth=6cm]{TODO}
%	  \includegraphics[width=1\linewidth]{"art/2016-07-27 21:21:12"}
	  \caption{Compound symmetry}  
%	\label{fig:2016-07-2721:21:12}
	\end{subfigure}%
	\begin{subfigure}{.5\textwidth}
	  \centering
	  \missingfigure[figwidth=6cm]{TODO}
%	  \includegraphics[width=1\linewidth]{"art/2016-07-27 11:42:05"}
	  \caption{AR(1)} %TODO
%	\label{fig:2016-07-2721:21:12}
	\end{subfigure}
\end{figure}



\begin{figure}[h]
\centering
\caption{\footnotesize [TODO].}	
\label{fig:large_sample}
	\begin{subfigure}{.5\textwidth}
	  \centering
	  \includegraphics[width=1\linewidth]{"art/2016-07-27 11:42:05zoom"}
	  \caption{n=40, zoom}  %TODO
	\label{fig:fig:large_sample_1}
	\end{subfigure}%
	\begin{subfigure}{.5\textwidth}
	  \centering
	  \includegraphics[width=1\linewidth]{"art/2016-08-04 13:59:33zoom"}
	  \caption{n=400, zoom} %TODO
	\label{fig:fig:large_sample_2}
	\end{subfigure}
\end{figure}




\begin{figure}[h]
\centering
\caption{\footnotesize [TODO].}	
%\label{fig:simulation_1}
	\begin{subfigure}{.5\textwidth}
	  \centering
	  	  \missingfigure[figwidth=6cm]{TODO}
%  \includegraphics[width=1\linewidth]{"art/2016-08-04 13:59:33"}
	  \caption{n=400, smaller effects}  %TODO
%	\label{fig:2016-07-2721:21:12}
	\end{subfigure}%
	\begin{subfigure}{.5\textwidth}
	  \centering
	  \missingfigure[figwidth=6cm]{TODO}
%	  \includegraphics[width=1\linewidth]{"art/2016-07-27 11:42:05"}
	  \caption{n=400, smaller effect, zoom} %TODO
%	\label{fig:2016-07-2721:21:12}
	\end{subfigure}
\end{figure}



\end{document}